



\begin{table*}
  \centering
  \small
  \begingroup
    \begin{tabularx}{\textwidth}{XXXXXXX}
      \toprule

      Condition
      & Increase in arousal
      & n, subjects in group
      & Most important attribute
      & 2nd most important attribute
      & 3rd most important attribute
      & 4th most important attribute
      \\

      \midrule


      Control
      & no
      & 13
      & \cellcontent{13.54\\(44.44)}
      & \cellcontent{12.77\\(48.48)}
      & \cellcontent{-5.54\\(35.50)}
      & \cellcontent{5.15\\(45.85)}
      \\

      & yes
      & 15
      & \cellcontent{14.47\\(46.96)}
      & \cellcontent{-1.8\\(50.26)}
      & \cellcontent{8.8\\(39.40)}
      & \cellcontent{-8.27\\(36.57)}
      \\

      Experiment
      & no
      & 12
      & \cellcontent{12.25\\(52.38)}
      & \cellcontent{12.92\\(51.86)}
      & \cellcontent{-10.5\\(48.02)}
      & \cellcontent{-14.5\\(37.87)}
      \\

      & yes
      & 16
      & \cellcontent{21.31\\(43.16)}
      & \cellcontent{6\\(33.09)}
      & \cellcontent{-7.25\\(36.38)}
      & \cellcontent{-7.88\\(29.36)}
      \\

      \bottomrule
    \end{tabularx}
    \endgroup
  \caption{Comparison of consolidation effect with respect to
    experimental condition and increase of arousal obtained from mood
    form. (Means above std deviations). Note. Differences are not
    significant (Tukey's HSD procedure, $\alpha=0.05$, $df = 3/52$,
    $3.40 < Q_{crit} < 3.41$, for all $Q_{obt}<0.047$).\label{tab:arousal}}
\end{table*}



